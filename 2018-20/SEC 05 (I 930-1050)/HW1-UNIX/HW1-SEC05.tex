\documentclass[letterpaper,10pt,onecolumn]{exam}
\usepackage[spanish]{babel}
\usepackage[utf8]{inputenc}
\usepackage{amsfonts}
\usepackage{amsthm}
\usepackage{amsmath}
\usepackage{mathrsfs}
%\usepackage{empheq}
\usepackage{enumitem}
\usepackage[pdftex]{color,graphicx}
\usepackage{hyperref}
\usepackage{listings}
\usepackage{calligra}
%\usepackage{algpseudocode} 
\DeclareMathAlphabet{\mathcalligra}{T1}{calligra}{m}{n}
\DeclareFontShape{T1}{calligra}{m}{n}{<->s*[2.2]callig15}{}
\newcommand{\scripty}[1]{\ensuremath{\mathcalligra{#1}}}
\lstloadlanguages{[5.2]Mathematica}
\setlength{\oddsidemargin}{0cm}
\setlength{\textwidth}{490pt}
\setlength{\topmargin}{-40pt}
\addtolength{\hoffset}{-0.3cm}
\addtolength{\textheight}{4cm}

%%%%%%%%%%%%%%%%%%%%%%%%%

\usepackage{marvosym} %use this pack
\usepackage{xcolor} %for color
%\usepackage{dtklogos}

%%%%%%%%%%%%%%%%%%%%%%%%%

\begin{document}
\begin{center}

\includegraphics[width=490pt]{header.png}\\[0.5cm]

\textsc{\huge Herramientas Computacionales}\\[0.1cm]

\Large \textsc{Semana 3. UNIX}\\
\Large \textsc{Sección-05}\\
\Large \textsc{2018-2}\\[0.7cm]

\end{center}


\noindent\rule{\textwidth}{1pt}\\[-0.1cm]

\newcounter{mysection}
\addtocounter{mysection}{1}

El archivo final (.sh) debe subirse a SicuaPlus con el nombre del estudiante en el formato \verb|NombreApellido_HW1.sh| antes que termine la clase.\\


En este taller vamos a trabajar con dos archivos que muestran los resultados de las pruebas ICFES para los años 2011 y 2012, con los cuales se busca explorar algunas herramientas de Linux para la edición y exploración de archivos de texto. Estos archivos se encuentran en los siguientes links:\\ 
\url{http://www.finiterank.com/saber/2012.csv}\\
\url{http://www.finiterank.com/saber/2011.csv}\\

Para el desarrollo del taller, se recomienda utilizar los comandos directamente en consola para comprobar su funcionamiento y luego consignarlos en el archivo \textit{.sh}. Cabe resaltar que al ejecutar el script este debe realizar cada uno de los puntos del taller, recuerde que puede crear archivos intermedios pero estos deben ser borrados después por el mismo script.

Por último, antes de la ejecución de cada punto el script debe indicar al usuario que punto está resolviendo con un mensaje de la forma \verb|Punto X...| donde \textit{X} corresponde al punto que esté resolviendo. Por ejemplo para el primer punto debe imprimir:

\centering \verb|Punto 1...|\\[0.7cm]

\begin{questions}
	\question[5] Cree una carpeta cuyo nombre corresponda a su nombre y apellido siguiendo el formato \verb|NombreApelido| y trabaje sobre ella.
	
	\question[5] Descargue los archivos con los resultados de las pruebas estatales.
	
	\question[10] En el archivo \verb|EncabezadosICFES.csv| (SicuaPlus) se encuentran los nombres de cada una de las columnas presentes en los archivos descargados en el punto anterior. Utilizando este archivo extraiga únicamente las columnas correspondientes a \verb|Puesto,Colegio, Departamento, Calendario, Promedio_Total,| \verb|Matemática, Física e Inglés| para cada uno de los archivos anteriores. Guarde este resultado en un archivo temporal.
	
	\question[10] Concatene los archivos anteriores en un único archivo llamado \verb|resultadosICFES2011-2012.csv|. Este archivo debe contener los resultados para en el año 2011 y luego continuar con los resultados del año 2012.
	
	\question[10] Imprima la cantidad de colegios registrados en las pruebas para cada uno de los años. El resultado debe ser de la forma:
	
	\verb|Colegios registrados en 2011:|\\
	\verb|n5|\\
	\verb|Colegios registrados en 2012:|\\
	\verb|n6|
	
	\question[10] Imprima el top 10 de colegios en ambas pruebas.
	
	\newpage
	
	\question[10] Imprima el número de colegios de calendario A en el top 10. El resultado debe ser de la forma:
	
	\verb|Total Colegios TOP10 A:|\\
	\verb|n8|\\
	
	\question[10] Imprima el número de colegios de calendario B en el top 10. El resultado debe ser de la forma:
	
	\verb|Total Colegios TOP10 B:|\\
	\verb|n9|\\
	
	\question[10] Imprima el número de colegios de calendario B y de Bogotá que presentaron las pruebas en 2011 y 2012. El resultado debe ser de la forma:
	
	\verb|Colegios Bogotá B:|\\
	\verb|n10|\\
	
	\question[10] Imprima el número de colegios en el top 10 de calendario A, cuyo promedio general sea mayor que el promedio de los resultados de Física y Matemáticas. El resultado debe ser de la forma:
	
	\verb|TOP10 A Pgen>PFisicaMatematicas:|\\
	\verb|n11|\\
	
	\question[10] Por último borre todos los archivos utilizados y la carpeta de trabajo. Recuerde no borrar el script en el que trabajó.	
	
	
\end{questions}	

\end{document}